\documentclass[]{report}

\voffset=-1.5cm
\oddsidemargin=0.0cm
\textwidth = 480pt

\usepackage{framed}
\usepackage{subfiles}
\usepackage{graphics}
\usepackage{newlfont}
\usepackage{eurosym}
\usepackage{amsmath,amsthm,amsfonts}
\usepackage{amsmath}
\usepackage{color}
\usepackage{enumerate}
\usepackage{amssymb}
\usepackage{multicol}
\usepackage[dvipsnames]{xcolor}
\usepackage{graphicx}
\begin{document}
%------------------------------------------------------- %

More tree terminology
A full binary tree.is a binary tree in which each node has exactly zero or two children.
A complete binary tree is a binary tree, which is completely filled, with the possible exception of the bottom level, which is filled from left to right.


\section{Spanning trees}
We say that a graph H is a subgraph of a graph G if

\begin{enumerate}
\item its vertices are
a subset of the vertex set of G, 
\item its edges are a subset of the edge set
of G, 
\item and each edge of H has the same end-vertices in G and H.
\end{enumerate} 

%------------------------------------------------------- %

More tree terminology
A full binary tree.is a binary tree in which each node has exactly zero or two children.
A complete binary tree is a binary tree, which is completely filled, with the possible exception of the bottom level, which is filled from left to right.


\section{Spanning trees}
We say that a graph H is a subgraph of a graph G if

\begin{enumerate}
\item its vertices are
a subset of the vertex set of G, 
\item its edges are a subset of the edge set
of G, 
\item and each edge of H has the same end-vertices in G and H.
\end{enumerate} 

%--------------------------------------------------------------------------%

\begin{description}
%Definition 3.3
\item[Definition]  If H is a subgraph of G such that V (H) = V (G), then H
is called a spanning subgraph of G. If H is a spanning subgraph
which is also a tree, then H is said to be a spanning tree of G.

\item[Example] In Figure 3.2, the graphs T1 and T2 are both spanning
trees of the graph G.
\end{description}


\begin{description}
%Definition 3.3
\item[Definition]  If H is a subgraph of G such that V (H) = V (G), then H
is called a spanning subgraph of G. If H is a spanning subgraph
which is also a tree, then H is said to be a spanning tree of G.

\item[Example] In Figure 3.2, the graphs T1 and T2 are both spanning
trees of the graph G.
\end{description}
\begin{itemize}
\item Node (or Vertex)
\item Key (or Record Number)
\item Root
\item Levels and Height
\item Parent and Child
\item Descendent and Ancestor
\item Subtrees
\item End Vertices (or External Nodes)
\end{itemize}


\end{document}
