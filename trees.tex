% %- https://www.tutorialspoint.com/graph_theory/index.htm




\section{Tree}
A connected acyclic graph is called a tree. In other words, a connected graph with no cycles is called a tree.

The edges of a tree are known as branches. Elements of trees are called their nodes. The nodes without child nodes are called leaf nodes.

A tree with ‘n’ vertices has ‘n-1’ edges. If it has one more edge extra than ‘n-1’, then the extra edge should obviously has to pair up with two vertices which leads to form a cycle. Then, it becomes a cyclic graph which is a violation for the tree graph.

Example 1
The graph shown here is a tree because it has no cycles and it is connected. It has four vertices and three edges, i.e., for ‘n’ vertices ‘n-1’ edges as mentioned in the definition.

Tree
Note − Every tree has at least two vertices of degree one.

Example 2
Tree 1
In the above example, the vertices ‘a’ and ‘d’ has degree one. And the other two vertices ‘b’ and ‘c’ has degree two. This is possible because for not forming a cycle, there should be at least two single edges anywhere in the graph. It is nothing but two edges with a degree of one.

Forest
A disconnected acyclic graph is called a forest. In other words, a disjoint collection of trees is called a forest.

Example
The following graph looks like two sub-graphs; but it is a single disconnected graph. There are no cycles in this graph. Hence, clearly it is a forest.

Forest
Spanning Trees
Let G be a connected graph, then the sub-graph H of G is called a spanning tree of G if −

H is a tree
H contains all vertices of G.
A spanning tree T of an undirected graph G is a subgraph that includes all of the vertices of G.

Example
Spanning Trees
In the above example, G is a connected graph and H is a sub-graph of G.

Clearly, the graph H has no cycles, it is a tree with six edges which is one less than the total number of vertices. Hence H is the Spanning tree of G.

\subsection{Circuit Rank}
Let ‘G’ be a connected graph with ‘n’ vertices and ‘m’ edges. A spanning tree ‘T’ of G contains (n-1) edges.

Therefore, the number of edges you need to delete from ‘G’ in order to get a spanning tree = m-(n-1), which is called the circuit rank of G.

This formula is true, because in a spanning tree you need to have ‘n-1’ edges. Out of ‘m’ edges, you need to keep ‘n–1’ edges in the graph.

Hence, deleting ‘n–1’ edges from ‘m’ gives the edges to be removed from the graph in order to get a spanning tree, which should not form a cycle.

Example
Take a look at the following graph −

Circuit Rank
For the graph given in the above example, you have m=7 edges and n=5 vertices.

Then the circuit rank is

\begin{eqnarray}
G = m – (n – 1)\\
  = 7 – (5 – 1)\\
  = 3\\
\end{eqnarray}

Example
Let ‘G’ be a connected graph with six vertices and the degree of each vertex is three. Find the circuit rank of ‘G’.

By the sum of degree of vertices theorem,

n
∑
i=1
 deg(Vi) = 2|E|
6 × 3 = 2|E|

|E| = 9

Circuit rank = |E| – (|V| – 1)

= 9 – (6 – 1) = 4

\subsection{Kirchoff’s Theorem}
Kirchoff’s theorem is useful in finding the number of spanning trees that can be formed from a connected graph.

Example
Kirchoff’s Theorem
The matrix ‘A’ be filled as, if there is an edge between two vertices, then it should be given as ‘1’, else ‘0’.

Kirchoff’s Theorem Example


\newpage



\end{document}
