
		

\section{Spanning Trees}
A spanning tree \textbf{T} of a connected, undirected graph \textbf{G} is a tree composed of all the vertices and some (or perhaps all) of the edges of \textbf{G}. 

Informally, a spanning tree of \textbf{G} is a selection of edges of \textbf{G} that form a tree spanning every vertex. That is, every vertex lies in the tree, but no cycles (or loops) are formed. On the other hand, every bridge of \textbf{G} must belong to \textbf{T}.

A spanning tree of a connected graph \textbf{G} can also be defined as a maximal set of edges of \textbf{G} that contains no cycle, or as a minimal set of edges that connect all vertices.


\section{Binary search tree (BST)}
a binary search tree (BST), sometimes also called an ordered or sorted binary tree, is a node-based binary tree data structure which has the following properties:

\begin{itemize}
\item The left subtree of a node contains only nodes with keys less than the node's key.
\item The right subtree of a node contains only nodes with keys greater than the node's key.
\item The left and right subtree must each also be a binary search tree.
\item There must be no duplicate nodes.
\end{itemize}
