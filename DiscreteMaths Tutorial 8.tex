\documentclass[12pt]{article}

\usepackage{amsmath}
\usepackage{amssymb}

%opening
\title{Mathematics for Computing}
%\author{Hibernia College}

\begin{document}

\begin{center}
\huge{Mathematics for Computing}\\
\LARGE{Trees}
\end{center}

\section{Trees}

Storing a list of records

Subtrees of a binary tree (8:3:1)


Example 8.2 Deleting a vertex from a graph

8.1.2 The number of edges in a tree.
Theorem 8.3 Let T be a tree with n vertices. Then T has n-1 edges.
%-----------------------
%8.1.3 Spanning Trees 
\subsection*{Spanning Trees}
Spanning Subgraph of G
Spanning Tree of G
8.2 Rooted Trees
%-------------------------------------------------------
%<page 37>
A balanced binary tree has $2^i$ nodes on all levels i apart from the highest level.

A tree in which one vertex has been singled out in this way

Let x and y be vertices of T. If the unique path from the root r
to x in T passes through y, then y is called an ancestor of x and x is a 
descendant of y.

If the vertices y and x are adjacent on the path from r to x, then
y is called the parent of x and x is called the child of y.

%-----------------------------
%< page 38 >
%Theorem 8.5

%8.2.1 Binary Trees
\subsection{Binary Trees}
A binary child is a rooted tree in which each internal node has exactly
two children, the left child and right child respectively.

%--------------------------------
< page 39 >
Balanced Binary Tree

Storing Data in a binary tree search(8:3:2)

%-----------------------------------
A tree is a connected graph that contains no cycles

A tree on n vertices has n-1 degrees






%------------------------------------------------%
\section*{Session 08 Trees}
\subsection*{Trees}
\begin{itemize}
\item 
\end{itemize}
%---------------------------------------------- %

\subsection*{Binary Search Trees}

\subsection*{Spanning Trees}


A spanning tree T of a connected, undirected graph G is a tree composed of all the vertices and some (or perhaps all) of the edges of G. Informally, a spanning tree of G is a selection of edges of G that form a tree spanning every vertex.


 That is, every vertex lies in the tree, but no cycles (or loops) are formed. On the other hand, every bridge of G must belong to T.
 
%------------------------------------------------------- %

\section{Introduction to Trees}
\begin{itemize}
\item What are trees?
\end{itemize}

%------------------------------------------------------- %

We extend the concept of linked data structures to structure containing nodes with more than one self-referenced field. A binary tree is made of nodes, where each node contains a "left" reference, a "right" reference, and a data element. The topmost node in the tree is called the root.

%------------------------------------------------------- %

Every node (excluding a root) in a tree is connected by a directed edge from exactly one other node. This node is called a parent. On the other hand, each node can be connected to arbitrary number of nodes, called children. Nodes with no children are called leaves, or external nodes. Nodes which are not leaves are called internal nodes. Nodes with the same parent are called siblings.

%------------------------------------------------------- %
   
More tree terminology:

The depth of a node is the number of edges from the root to the node.
The height of a node is the number of edges from the node to the deepest leaf.
The height of a tree is a height of the root.



%------------------------------------------------------- %
\end{document}
